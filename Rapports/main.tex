\documentclass{article}
\usepackage[font=small,labelfont=bf]{caption}
\usepackage{mathrsfs}
\usepackage[utf8]{inputenc}
\usepackage{amsmath, fge}
\usepackage{amssymb}
\usepackage{amsthm}
\usepackage[margin=1in]{geometry}
\usepackage{amsfonts}
\usepackage{upgreek}
\usepackage{dashbox}
\usepackage[bottom]{footmisc}

\newcommand\dboxed[1]{\dbox{\ensuremath{#1}}}

\usepackage{mathtools}
\newlength{\temp}
\usepackage{siunitx}
\usepackage{hyperref}
\usepackage{enumitem}
\usepackage{amsbsy}
\hypersetup{
    colorlinks=true,
    linkcolor=blue,
    filecolor=magenta,      
    urlcolor=cyan,
    pdftitle={Overleaf Example},
    pdfpagemode=FullScreen,
    }
\usepackage[french]{babel}
\usepackage{mleftright,mathtools}
\usepackage{bm}
\usepackage{pdfpages}
\AtBeginDocument{
  \DeclareSymbolFont{AMSb}{U}{msb}{m}{n}
  \DeclareSymbolFontAlphabet{\mathbb}{AMSb}
  }
\usepackage{fancyhdr}
\pagestyle{fancy}
\lhead{Ahou L. -  Ahou S. - Fiorini L. - Portier A.} % controls the left corner of the header
\chead{} % controls the center of the header
\rhead{Groupe 49} % controls the right corner of the header
\lfoot{} % controls the left corner of the footer
\cfoot{} % controls the center of the footer
\cfoot{~\thepage} % controls the right corner of the footer
\usepackage{titlesec}
\renewcommand{\headrulewidth}{0.4pt}
\renewcommand{\footrulewidth}{0.4pt}
\usepackage{letltxmacro}

\newcommand{\dx}{\mathrm{d}x}
\newcommand{\dr}{\mathrm{d}r}
\newcommand{\dt}{\mathrm{d}t}
\LetLtxMacro{\OldSqrt}{\sqrt}
\newcommand{\ClosedSqrt}[1][\hphantom{3}]{\def\DHLindex{#1}\mathpalette\DHLhksqrt}
\makeatletter
    \newcommand*\bold@name{bold}
    \def\DHLhksqrt#1#2{%
        \setbox0=\hbox{$#1\OldSqrt{#2\,}$}\dimen0=\ht0\relax%
        \advance\dimen0-0.2\ht0\relax% size of the added box is still 0.2 times ht0
        \setbox2=\hbox{\vrule height\ht0 depth -\dimen0}%
        {\hbox{$#1\expandafter\OldSqrt\expandafter[\DHLindex]{#2\,}$}
        \lower\ifx\math@version\bold@name0.6pt\else0.4pt\fi\box2}
    }
    % root index positioning and added space at the end, mostly noticeable in inline math mode
    \renewcommand*{\sqrt}[2][\ ]{\ClosedSqrt[\leftroot{-2}\uproot{1}#1]{#2}\kern0.1em} 
\makeatother
\newcommand{\Chi}{\raisebox{0.25pt}[1ex][0.75ex]{$\chi$}}
\usepackage{bigints}
\newcommand{\R}{\mathbb{R}}
\newcommand{\N}{\mathbb{N}}

\thispagestyle{plain}
\titlelabel{\thetitle \quad}



\begin{document}


%\pagestyle{fancy}
\thispagestyle{empty}
\begin{titlepage}
	\begin{center}
		\begin{figure}
			\raisebox{-0.5\height}{\includegraphics[width=.35\textwidth]{logo_ucl.png}}\hfill
			\raisebox{-0.5\height}{\includegraphics[width=.1\textwidth]{logo_epl.png}}
        \end{figure}
		
		\textsc{\LARGE École Polytechnique de Louvain\\faculté de l'Université catholique de Louvain}\\[1cm]

		\textsc{\large LINMA1702 - Modèles et méthodes d'optimisation I}\\[0.5cm]
		\textsc{\large Année académique 2023 -- 2024}\\[0.5cm]

		% Title
		\HRule \\[0.4cm]
		{ \huge \bfseries Rapport : 1\textsuperscript{ère} partie\\[0.4cm] }
		\HRule \\[0.75cm]

		% Author and supervisor
		
\begin{minipage}{0.4\textwidth}
\begin{flushleft}
\Large
Groupe 49:\\
\textsc{Ahou} Lucas\\
\textsc{Ahou}  Samuel \\
\textsc{Fiorini} Lucien \\
\textsc{Portier}  Adrien \\

\end{flushleft}
\end{minipage}
\begin{minipage}{0.4\textwidth}
\begin{flushright}
\Large
NOMA:\\
3594-22-00\\
4408-19-00\\
7502-22-00 \\
5337-22-00\\
\end{flushright}
\end{minipage}


\\ [2 cm]
\Large
Professeur:\\
\textsc{Glineur} François\\

		
		\vfill
		%\begin {center}
		%    \includegraphics[width=0.45\textwidth]{img/couv1}
		%\end{center}
		
		\vfill
		
		% Bottom of the page
		{\large Avril 2024}
		
		\newpage
		%\thispagestyle{empty}

		\setcounter{tocdepth}{3}
		\vfill
	\end{center}
\end{titlepage}
%-----------------------------------------------------------
% début du document numéroté
\clearpage
\pagenumbering{arabic}

\section*{Question 1}

\subsection*{A.}
\subsubsection*{Modélisation du problème}
Afin de garantir que le niveau de production ne soit pas trop bas, il nous a été demandé de maximiser le niveau minimal de production au cours de l'année. Nous allons discuter dans cette section de notre choix de modélisation pour ce problème.\\
Vous trouverez ci-dessous une table regroupant les notations utilisées dans notre modèle \footnote{Bien que $n$ et $m$ soient des constantes, nous utilisons ces notations pour alléger les formules et également pour rester dans un cadre général}:
\begin{table}[h!]
\centering
\renewcommand{\arraystretch}{1.5}% Add spacing between rows : default value is 1
\begin{tabular}{|c || c |} 
 \hline
Nom & Signification\\
 \hline\hline
 $P$  & Puissance totale à installer\\
 $\kappa \in [0, 1]$ & Proportion de la puissance totale à installer en sites \textit{offshore}\\
 $c^\text{max}_i$ & Capacité maximale installable pour le i\textsuperscript{ème} site\\
 $c_i \in [0, c_i^{\max}]$ & Capacité effectivement installée sur le i\textsuperscript{ème} site\\
 $e_i(t) \in [0,1]$ & Rendement du i\textsuperscript{ème} site au temps $t$\\
 $n$ & Nombre de sites\\
 $m$ & Nombre d'heures dans une année\\
 \hline
\end{tabular}
\caption{Table des notations utilisées pour le modèle}
\label{table:noms_des_variables}
\end{table}
\noindent \\
Parmi toutes ces variables, ce sont les $c_i$ que nous cherchons à optimiser. L'idée derrière notre choix de modèle est la suivante : \\
Pour toutes les heures de l'année, nous prenons la somme de production de tous les sites en tenant compte du choix des $c_i$. Nous obtenons ainsi une fonction de la production totale en fonction du temps et dont l'allure dépend du choix des $c_i$.
Un tableau illustratif est repris ci-dessous :
\begin{table}[h!]
\centering
    \[ \begin{array}{c|cccc}
      t_i    & {t_0} & {t_1} & {\ldots}  & {t_{m-1}}\\
      \hline
      e_0(t_i) & 0.25       & 0.08   &\ldots    & 0.17 \\
      e_1(t_i) & 0.87       & 0.69   &\ldots    & 0.42 \\
      \vdots   & \vdots     & \vdots  &\vdots   & \vdots \\
      e_{n-1}(t_i) & 0.15       & 0.207   &\ldots    & 0.94 \\
      \hline
      \text{Prod. totale} &\scriptstyle \sum \limits_{i=0}^{n-1}c_i e_i(t_0) &\ldots &\ldots &\scriptstyle \sum \limits_{i=0}^{n-1}c_i e_i(t_{m-1})\\
    \end{array} \]
\caption{Table représentant les valeurs de rendement de chaque site ainsi que la production totale en fonction du temps (à titre illustratif).}
\label{table:table_rendement_illustratif}
\end{table}
\newpage
Nous cherchons donc à maximiser la valeur minimale de la production totale. Avec toutes ces informations, nous pouvons écrire notre modèle :
\begin{align*}
    \max_{c_i} \quad  
    \min_t \{ &\sum_{i=0}^{n-1}c_i e_i(t) \} \\ 
    \sum_{i=0}^{n-1} c_i &= P\\
    \sum_{\text{\tiny offshore}}c_i &= \kappa P\\
    0 &\le \, c_i \, \le c_i^{\max}\\
\end{align*}
Cependant, celui-ci n'est pas linéaire à cause de la fonction $\min$. Nous introduisons alors une variable intermédiaire $\gamma$ afin de remplacer le $\min$ par un problème de maximisation comme vu au cours. Le nouveau modèle, maintenant linéaire, s'écrit comme suit \footnote{La contrainte imposant que la variable $\gamma$ soit positive n'est pas nécessaire car le minimum est strictement positif et que $\gamma$ prendra cette valeur.} :
\begin{align}
    \max_{c_i, \gamma} \quad  
    &\gamma \nonumber \\ 
    \sum_{i=0}^{n-1} c_i &= P \label{eq:contr_1}\\
    \sum_{\text{\tiny offshore}}c_i &= \kappa P \label{eq:contr_2}\\
    \gamma &\le \sum_{i=0}^{n-1} c_i e_i(t_j) \quad \forall j \in \{0, \ldots, m-1\}\label{eq:contr_3}\\
    0 &\le \, c_i \, \le c_i^{\max} \quad \forall i \in \{0, \ldots, n-1\} \label{eq:contr_4}
\end{align}
La contrainte (\ref{eq:contr_1}) représente la puissance totale qu'il faut installer en Europe. La contrainte (\ref{eq:contr_2}) indique qu'il faut exactement une proportion $\kappa$ de la puissance $P$ installée en \textit{offshore}. La contrainte (\ref{eq:contr_3}) dit que la variable intermédiaire $\gamma$ ne doit pas dépasser la valeur du minimum de la fonction de production totale. Enfin, la contrainte (\ref{eq:contr_4}) indique les bornes sur les variables $c_i$

\subsubsection*{Estimation de la taille du modèle}
\noindent
Au total, nous avons $\overbrace{n}^{c_i} + \overbrace{1}^{\gamma}$
variables et $\overbrace{m}^{\text{(3)}} + \overbrace{2}^{\text{(1) \& (2)}}$ contraintes \textbf{en plus} des bornes sur les $c_i$.\\
Cependant, lors de la résolution le \textit{solver} traduit le problème sous forme standard. Ainsi, il introduit $n + m$ variables de \textit{slack} qui correspondent au contraintes (\ref{eq:contr_3}) et (\ref{eq:contr_4}). Nous nous retrouvons alors avec le problème suivant : 
\begin{align*}
    \max_{c_i, \gamma} \quad  
    &\gamma \nonumber \\ 
    \sum_{i=0}^{n-1} c_i &= P\\
    \sum_{\text{\tiny offshore}}c_i &= \kappa P\\
    \gamma + s_j &= \sum_{i=0}^{n-1} c_i e_i(t_j) \quad \forall j \in \{0, \ldots, m-1\}\\
    c_i + t_i &\le c_i^{\max} \quad \forall i \in \{0, \ldots, n-1\}\\
    c_i, \gamma, s_j, t_i &\ge 0
\end{align*}
Ce qui mène à un total de $2n + m + 1$ variables et $m + n + 2$ contraintes.

\subsection*{B.}
Pour résoudre le modèle décrit dans la section précédente, nous avons utilisé la fonction \verb|linprog| de la librairie \verb|SciPy|.\\
La solution retournée par le solver pour les paramètres $P = 500000 [\mathrm{MW}]$ et $\kappa = 0.17$ nous indique qu'il y a $267$ sites sur lesquels la puissance installée est non nulle, ce qui veut dire qu'il y a $642 - 267 = 375$ sites sur lesquels il est préférable de ne pas installer d'éoliennes.\\
La répartition de puissance est représentée sur la carte ci-dessous \footnote{Les croix représentent les sites sur lesquels la puissance installée est nulle.} :

\begin{figure}[h!]
    \centering
    \includegraphics[scale=0.5]{Images/Partie_1/capacity_distribution.pdf}
    \caption{Carte représentant la répartition de puissance sur les différents sites d'éoliennes}
    \label{fig:capacity_distribution_partie1}
\end{figure}

\newpage

Quant au temps de résolution du solver, celui-ci prend en moyenne $1.825 [\mathrm{s}]$ pour renvoyer une solution. Un graphe des performances est représenté ci-dessous :

\begin{figure}[h!]
    \centering
    \includegraphics[scale=0.5]{Images/Partie_1/execution_time.pdf}
    \caption{Graphe du temps d'exécution sur 100 itérations}
    \label{fig:execution_time_partie1}
\end{figure}

Nous pouvons également nous intéresser à l'énergie totale produite sur l'année ainsi que sur chaque période d'une heure.
En prenant compte des rendements qui nous ont été fournis et en utilisant la solution obtenue par le solver,
il est possible de calculer, en chaque heure, l'énergie produite. Le graphe représentant la production d'énergie en fonction du temps est représenté ci-dessous :

\begin{figure}[h!]
    \centering
    \includegraphics[scale=0.5]{Images/Partie_1/energy_produced.pdf}
    \caption{Graphe de la production d'énergie en fonction du temps}
    \label{fig:energy_produced_partie1}
\end{figure}
Et en sommant toutes les productions, nous obtenons une valeur d'environ $1828762.56 [\mathrm{GWh}]$. 
\newpage
Nous pouvons, de manière similaire, calculer le rendement de production simplement en divisant la production sur une heure par la production idéale (i.e. $P*1\mathrm{h} = P [\mathrm{MWh}]$).\\
Nous obtenons alors le graphe suivant où nous pouvons observer une moyenne de $42\%$ de rendement :

\begin{figure}[h!]
    \centering
    \includegraphics[scale=0.5]{Images/Partie_1/efficiency.pdf}
    \caption{Graphe du rendement de production en fonction du temps}
    \label{fig:efficiency_partie1}
\end{figure}

\subsection*{C.}
\subsubsection*{(a)}
Nous avons vu au cours que les variables du problème dual associé au primal permettent de calculer la variation de la fonction objectif
lorsque l'on modifie la valeur d'une des contraintes. Lorsque l'on change la valeur de la puissance maximale installable de $P$ à $\Delta P$, la variable $y_1$ du problème dual nous indiquera de combien change
la fonction objectif (i.e. la valeur de production minimale). 
\end{document} 